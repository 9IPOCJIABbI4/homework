
\documentclass[a4paper,12pt]{article}

% В этом документе преамбула

%%% Работа с русским языком
\usepackage{cmap}					% поиск в PDF
\usepackage{mathtext} 				% русские буквы в формулах
\usepackage[T2A]{fontenc}			% кодировка
\usepackage[utf8x]{inputenc}			% кодировка исходного текста
\usepackage[english,russian]{babel}	% локализация и переносы
\usepackage{indentfirst}
\frenchspacing

\renewcommand{\epsilon}{\ensuremath{\varepsilon}}
\renewcommand{\phi}{\ensuremath{\varphi}}
\renewcommand{\kappa}{\ensuremath{\varkappa}}
\renewcommand{\le}{\ensuremath{\leqslant}}
\renewcommand{\leq}{\ensuremath{\leqslant}}
\renewcommand{\ge}{\ensuremath{\geqslant}}
\renewcommand{\geq}{\ensuremath{\geqslant}}
\renewcommand{\emptyset}{\varnothing}

%%% Дополнительная работа с математикой
\usepackage{amsmath,amsfonts,amssymb,amsthm,mathtools} % AMS
\usepackage{icomma} % "Умная" запятая: $0,2$ --- число, $0, 2$ --- перечисление

%% Номера формул
%\mathtoolsset{showonlyrefs=true} % Показывать номера только у тех формул, на которые есть \eqref{} в тексте.
%\usepackage{leqno} % Нумереация формул слева

%% Свои команды
\DeclareMathOperator{\sgn}{\mathop{sgn}}

%% Перенос знаков в формулах (по Львовскому)
\newcommand*{\hm}[1]{#1\nobreak\discretionary{}
{\hbox{$\mathsurround=0pt #1$}}{}}

%%% Работа с картинками
\usepackage{graphicx}  % Для вставки рисунков
\graphicspath{{images/}{images2/}}  % папки с картинками
\setlength\fboxsep{3pt} % Отступ рамки \fbox{} от рисунка
\setlength\fboxrule{1pt} % Толщина линий рамки \fbox{}
\usepackage{wrapfig} % Обтекание рисунков текстом

%%% Работа с таблицами
\usepackage{array,tabularx,tabulary,booktabs} % Дополнительная работа с таблицами
\usepackage{longtable}  % Длинные таблицы
\usepackage{multirow} % Слияние строк в таблице

%%% Теоремы
\theoremstyle{plain} % Это стиль по умолчанию, его можно не переопределять.
\newtheorem{theorem}{Теорема}[section]
\newtheorem{proposition}[theorem]{Утверждение}
 
\theoremstyle{definition}% "Определение"
\newtheorem*{dfn}{Определение}
\newtheorem{corollary}{Следствие}[theorem]
\newtheorem{problem}{Задача}[section]
 
\theoremstyle{remark} % "Примечание"
\newtheorem*{nonum}{Решение}

%%% Программирование
\usepackage{etoolbox} % логические операторы

%%% Страница
\usepackage{extsizes} % Возможность сделать 14-й шрифт
\usepackage{geometry} % Простой способ задавать поля
	\geometry{top=25mm}
	\geometry{bottom=35mm}
	\geometry{left=35mm}
	\geometry{right=20mm}
 %
%\usepackage{fancyhdr} % Колонтитулы
% 	\pagestyle{fancy}
 	%\renewcommand{\headrulewidth}{0pt}  % Толщина линейки, отчеркивающей верхний колонтитул
% 	\lfoot{Нижний левый}
% 	\rfoot{Нижний правый}
% 	\rhead{Верхний правый}
% 	\chead{Верхний в центре}
% 	\lhead{Верхний левый}
%	\cfoot{Нижний в центре} % По умолчанию здесь номер страницы

\usepackage{setspace} % Интерлиньяж
%\onehalfspacing % Интерлиньяж 1.5
%\doublespacing % Интерлиньяж 2
%\singlespacing % Интерлиньяж 1

\usepackage{lastpage} % Узнать, сколько всего страниц в документе.

\usepackage{soul} % Модификаторы начертания

\usepackage{hyperref}
\usepackage[usenames,dvipsnames,svgnames,table,rgb]{xcolor}
\hypersetup{				% Гиперссылки
    unicode=true,           % русские буквы в раздела PDF
    pdftitle={Заголовок},   % Заголовок
    pdfauthor={Автор},      % Автор
    pdfsubject={Тема},      % Тема
    pdfcreator={Создатель}, % Создатель
    pdfproducer={Производитель}, % Производитель
    pdfkeywords={keyword1} {key2} {key3}, % Ключевые слова
    colorlinks=true,       	% false: ссылки в рамках; true: цветные ссылки
    linkcolor=red,          % внутренние ссылки
    citecolor=black,        % на библиографию
    filecolor=magenta,      % на файлы
    urlcolor=cyan           % на URL
}

\usepackage{csquotes} % Еще инструменты для ссылок

%\usepackage[style=apa,maxcitenames=2,backend=biber,sorting=nty]{biblatex}

\usepackage{multicol} % Несколько колонок

\usepackage{tikz} % Работа с графикой
\usepackage{pgfplots}
\usepackage{pgfplotstable}
%\usepackage{coloremoji}
\usepackage{floatrow}
\usepackage{subcaption}
\newcommand*{\N}{\mathbb{N}}
\newcommand*{\R}{\mathbb{R}}
\newcommand*{\K}{\mathbb{K}}
\newcommand*{\V}{\mathcal{V}}
\newcommand*{\A}{\mathcal{A}}
\newcommand*{\ii}{\mathbf{1}}
\newcommand*{\oo}{\mathbf{0}}
\newcommand*{\ba}{\mathbf{a}}
\newcommand*{\bb}{\mathbf{b}}
\newcommand*{\Q}{\mathbb{Q}}
\newcommand*{\CC}{\mathbb{C}}
\graphicspath{{images/}}
%\setcounter{secnumdepth}{0}
%\addbibresource{master.bib}
\usepackage{titlesec}% http://ctan.org/pkg/titlesec
\titleformat{\section}{\normalfont\Large\bfseries}{Часть }{}{\thesection}
%\renewcommand{\thesection}{}% Remove section references\ldots
\renewcommand{\thesubsection}{\arabic{subsection}}%... from subsections
\usepackage{marginnote}



%%% Local Variables:
%%% mode: latex
%%% TeX-master: "master"
%%% End:


%\author{Ярослав😎}
\title{Домашка по теории групп}
%\date{\today}

%\includeonly{chapters/ch2,chapters/ch3}

\begin{document} % конец преамбулы, начало документа

\maketitle

\section{}
\subsection{$\mathcal{U}_\mathbf{H}$ }
\begin{dfn}
	\emph{Группой} $\,\mathcal{U}_\mathbf{H}$ называется группа невырожденных
	комплексных матриц $T$, удовлетворяющих соотношению
	\begin{equation}
		\mathbf{H}=T^\dag \cdot \mathbf{H} \cdot T
		\label{eq:1}
	,\end{equation}
	где $\mathbf{H}$ --- некоторая эрмитова матрица.
\end{dfn}
\subsection{$O(n,\, \CC )$ }
\begin{dfn}
	\emph{Ортогональной группой $O(n,\,\CC)$} называется группа комплексных
	ортогональных
	матриц
	\begin{equation}
	O^T \cdot O = I_n
	\label{eq:2}
.\end{equation}
\end{dfn}
\subsection{$SO(n,\, \CC )$ }
\begin{dfn}
	\emph{Специальной ортогональной группой $SO(n,\,\CC)$} называется
	подгруппа комплексных ортогональных
матриц, удовлетворяющих (\ref{eq:2}), и таких, что $\det (O)=+1$.
\end{dfn}
\subsection{$Sp(2r,\, \R)$ }
\begin{dfn}
	\emph{Симплектической группой $Sp(2r,\, \R)$ } называется группа
	вещественных матриц $T$,
		удовлетворяющих условию симплектичности
	\begin{equation}
		T^T \cdot \begin{pmatrix} 0 & I_{r} \\ -I_{r} & 0 \end{pmatrix} 
		\cdot T= \begin{pmatrix} 0 & I_{r} \\ -I_{r} & 0 \end{pmatrix} 
		\label{eq:3}
	.\end{equation}

\end{dfn}
\subsection{$Sp(p,\, q)$ }
\begin{dfn}
	\emph{ Группой $Sp(p,\,q)$}, где $p+q=r$, называется группа
	комплексных $2r \times 2r$ матриц $T$, удовлетворяющих одновременно
	условию псевдо-унитарности
	\begin{equation}
	T^\dag \cdot \begin{pmatrix} 
	I_{p,\, q} & 0\\
0 & I_{p,\,q}
\end{pmatrix} \cdot T= \begin{pmatrix}
I_{p,\,q} & 0\\
0 & I_{p,q}\end{pmatrix} 
	\end{equation}
	и нестандартному условию симплектичности
	\begin{equation}
		T^T \cdot \begin{pmatrix} 0 & I_{p,\,q} \\ -I_{p,\,q} & 0 \end{pmatrix} 
		\cdot T= \begin{pmatrix} 0 & I_{p,\,q} \\ -I_{p,\,q} & 0 \end{pmatrix} 
	.\end{equation}
\end{dfn}

\subsection{$USp(2r)$ }
\begin{dfn}
	\emph{Унитарной симлектической группой $USP(2r)$} называется
	группа комплексных $2r \times 2r$ матриц $T$, удовлетворяющих как условию
	симплектичности  (\ref{eq:3}), так и условию унитарности $T^\dag \cdot
	T=I_n$.
\end{dfn}
\subsection{$O(p, q)$ }
\begin{dfn}
	\emph{Вещественной псевдо-ортогонально группой $O(p,\,q)$} называется
	группа вещественных $(n \times n)$ матриц $O$, где $n=p+q$, подчиняющихся
	условию
	\begin{equation}
		O^T \cdot
		\begin{pmatrix} 
		I_p & 0 \\
	0 & -I_q
\end{pmatrix}
\cdot O = 
\begin{pmatrix} 
I_p & 0 \\
0 & -I_q\end{pmatrix} 
	.
	\label{eq:6}
	\end{equation}
\end{dfn}
\subsection{$PSO(p, q)$ }
\begin{dfn}
	\emph{Специальной вещественной псевдо-ортогональной группой $SO(p,\,q)$}
	называется
	группа вещественных $(n \times n)$ матриц $O$, где $n=p+q$, подчиняющихся
	одноврменно условию псевдоортогональности (\ref{eq:6}), а также
	условию $\det O =1$.
\end{dfn}

\begin{dfn}
	\emph{Проективной псевдо-ортогональной группой $PSO(p,\,q)$}
	называется фактор-группа $SO(p,\,q) / \mathbf{Z}_2$.
\end{dfn}
\subsection{Группа Лоренца}
\begin{dfn}
	\emph{Группой Лоренца $n$-мерного пространства}
	называется группа $O(1,n-1)$.
\end{dfn}
\subsection{$SU(p, q)$ }
\begin{dfn}
	\emph{Специальной псевдо унитарной группой $SU(p,\,q)$} называется
	группа комплексных $(n \times n)$ матриц $U$, удовлетворяющих
	одновременно условию (\ref{eq:1}),
	где в качестве $\mathbf{H}$ выбрана матрица $I_{p,\,q}$:
\begin{equation}
	U^\dag  \cdot \begin{pmatrix} I_{p,\,q} & 0 \\ 0 & I_{p,\,q}
	\end{pmatrix} \cdot U= \begin{pmatrix} I_{p,\,q} & 0 \\  0 & I_{p,\,q}
\end{pmatrix} 
\end{equation}
и условию  $\det U =1$.
\end{dfn}
\emph{Центром группы} является конечная инвариантная подгруппа матриц
\begin{equation}
	U_k =\exp \left( i \frac{2 \pi k}{n} \right) I_n \quad (k = 0, 1, \ldots,
	n-1)
.\end{equation}
\subsection{$PSL(n, \K)$ }
\begin{dfn}
	\emph{Проективной комплексной линейной группой $PSL(n,\,\CC)$}
	называется фактор-группа $SL(n,\,\CC) / \mathbf{Z}_n$.
\end{dfn}
\begin{dfn}
	\emph{Проективной вещественной линейной группой $PSL(2n,\,\R$}
	называется фактор-группа $SL(2n\,\R)/\mathbf{Z}_2$.
\end{dfn}
\subsection{$\mathcal{O}_{I_n}$ }
$\mathcal{O}_{I_n} \equiv O(n,\,\CC)$
\subsection{Группа анти-де Ситтера}
\begin{dfn}
	\emph{Группой анти-де Ситтера $n$-мерного пространства} называется
	группа $O(2,n-2)$.
\end{dfn}
\subsection{$PSU(p, q)$ }
\begin{dfn}
	\emph{Проективной псевдо-унитарной группой $PSU(p,\,q)$} называется
	фактор-группа $SU(p,\,q) / \mathbf{Z}_n$, где  $p+q=n$.
\end{dfn}
\subsection{$\mathcal{O}_{J}$ }
$\mathcal{O}_{J} \equiv Sp(2r,\, \CC)$
\subsection{Группа Пуанкаре}
\begin{dfn}
	\emph{Группой Пункаре} называется группа симметрий четырёхмерного
	пространства Минковского, т.е. множество преобразований вида
	\begin{equation}
		x_k \to x'_k = O_{kj}x_j+a_k
	,\end{equation}
	где вещественная матрица $\|O_{ij}\| \in O(1,\,3)$ и произведение
	элементов задаётся следующим образом
	\begin{equation}
		g_1 \cdot g_2 = (O_1, \vec{a}_1) \cdot  (O_2, \vec{a}_2) =
		(O_1 \cdot O_2,\, O_1 \cdot \vec{a}_2 + \vec{a}_1)
	.\end{equation}
\end{dfn}
\end{document} % конец документа

