
\documentclass[a4paper,12pt]{article}

\input{preamble}

%\author{Ярослав😎}
\title{Домашка по теории групп}
%\date{\today}

%\includeonly{chapters/ch2,chapters/ch3}

\begin{document} % конец преамбулы, начало документа

\maketitle

\section{}
\subsection{$\mathcal{U}_\mathbf{H}$ }
\begin{dfn}
	\emph{Группой} $\,\mathcal{U}_\mathbf{H}$ называется группа невырожденных
	комплексных матриц $T$, удовлетворяющих соотношению
	\begin{equation}
		\mathbf{H}=T^\dag \cdot \mathbf{H} \cdot T
		\label{eq:1}
	,\end{equation}
	где $\mathbf{H}$ --- некоторая эрмитова матрица.
\end{dfn}
\subsection{$O(n,\, \CC )$ }
\begin{dfn}
	\emph{Ортогональной группой $O(n,\,\CC)$} называется группа комплексных
	ортогональных
	матриц
	\begin{equation}
	O^T \cdot O = I_n
	\label{eq:2}
.\end{equation}
\end{dfn}
\subsection{$SO(n,\, \CC )$ }
\begin{dfn}
	\emph{Специальной ортогональной группой $SO(n,\,\CC)$} называется
	подгруппа комплексных ортогональных
матриц, удовлетворяющих (\ref{eq:2}), и таких, что $\det (O)=+1$.
\end{dfn}
\subsection{$Sp(2r,\, \R)$ }
\begin{dfn}
	\emph{Симплектической группой $Sp(2r,\, \R)$ } называется группа
	вещественных матриц $T$,
		удовлетворяющих условию симплектичности
	\begin{equation}
		T^T \cdot \begin{pmatrix} 0 & I_{r} \\ -I_{r} & 0 \end{pmatrix} 
		\cdot T= \begin{pmatrix} 0 & I_{r} \\ -I_{r} & 0 \end{pmatrix} 
		\label{eq:3}
	.\end{equation}

\end{dfn}
\subsection{$Sp(p,\, q)$ }
\begin{dfn}
	\emph{ Группой $Sp(p,\,q)$}, где $p+q=r$, называется группа
	комплексных $2r \times 2r$ матриц $T$, удовлетворяющих одновременно
	условию псевдо-унитарности
	\begin{equation}
	T^\dag \cdot \begin{pmatrix} 
	I_{p,\, q} & 0\\
0 & I_{p,\,q}
\end{pmatrix} \cdot T= \begin{pmatrix}
I_{p,\,q} & 0\\
0 & I_{p,q}\end{pmatrix} 
	\end{equation}
	и нестандартному условию симплектичности
	\begin{equation}
		T^T \cdot \begin{pmatrix} 0 & I_{p,\,q} \\ -I_{p,\,q} & 0 \end{pmatrix} 
		\cdot T= \begin{pmatrix} 0 & I_{p,\,q} \\ -I_{p,\,q} & 0 \end{pmatrix} 
	.\end{equation}
\end{dfn}

\subsection{$USp(2r)$ }
\begin{dfn}
	\emph{Унитарной симлектической группой $USP(2r)$} называется
	группа комплексных $2r \times 2r$ матриц $T$, удовлетворяющих как условию
	симплектичности  (\ref{eq:3}), так и условию унитарности $T^\dag \cdot
	T=I_n$.
\end{dfn}
\subsection{$O(p, q)$ }
\begin{dfn}
	\emph{Вещественной псевдо-ортогонально группой $O(p,\,q)$} называется
	группа вещественных $(n \times n)$ матриц $O$, где $n=p+q$, подчиняющихся
	условию
	\begin{equation}
		O^T \cdot
		\begin{pmatrix} 
		I_p & 0 \\
	0 & -I_q
\end{pmatrix}
\cdot O = 
\begin{pmatrix} 
I_p & 0 \\
0 & -I_q\end{pmatrix} 
	.
	\label{eq:6}
	\end{equation}
\end{dfn}
\subsection{$PSO(p, q)$ }
\begin{dfn}
	\emph{Специальной вещественной псевдо-ортогональной группой $SO(p,\,q)$}
	называется
	группа вещественных $(n \times n)$ матриц $O$, где $n=p+q$, подчиняющихся
	одноврменно условию псевдоортогональности (\ref{eq:6}), а также
	условию $\det O =1$.
\end{dfn}

\begin{dfn}
	\emph{Проективной псевдо-ортогональной группой $PSO(p,\,q)$}
	называется фактор-группа $SO(p,\,q) / \mathbf{Z}_2$.
\end{dfn}
\subsection{Группа Лоренца}
\begin{dfn}
	\emph{Группой Лоренца $n$-мерного пространства}
	называется группа $O(1,n-1)$.
\end{dfn}
\subsection{$SU(p, q)$ }
\begin{dfn}
	\emph{Специальной псевдо унитарной группой $SU(p,\,q)$} называется
	группа комплексных $(n \times n)$ матриц $U$, удовлетворяющих
	одновременно условию (\ref{eq:1}),
	где в качестве $\mathbf{H}$ выбрана матрица $I_{p,\,q}$:
\begin{equation}
	U^\dag  \cdot \begin{pmatrix} I_{p,\,q} & 0 \\ 0 & I_{p,\,q}
	\end{pmatrix} \cdot U= \begin{pmatrix} I_{p,\,q} & 0 \\  0 & I_{p,\,q}
\end{pmatrix} 
\end{equation}
и условию  $\det U =1$.
\end{dfn}
\emph{Центром группы} является конечная инвариантная подгруппа матриц
\begin{equation}
	U_k =\exp \left( i \frac{2 \pi k}{n} \right) I_n \quad (k = 0, 1, \ldots,
	n-1)
.\end{equation}
\subsection{$PSL(n, \K)$ }
\begin{dfn}
	\emph{Проективной комплексной линейной группой $PSL(n,\,\CC)$}
	называется фактор-группа $SL(n,\,\CC) / \mathbf{Z}_n$.
\end{dfn}
\begin{dfn}
	\emph{Проективной вещественной линейной группой $PSL(2n,\,\R$}
	называется фактор-группа $SL(2n\,\R)/\mathbf{Z}_2$.
\end{dfn}
\subsection{$\mathcal{O}_{I_n}$ }
$\mathcal{O}_{I_n} \equiv O(n,\,\CC)$
\subsection{Группа анти-де Ситтера}
\begin{dfn}
	\emph{Группой анти-де Ситтера $n$-мерного пространства} называется
	группа $O(2,n-2)$.
\end{dfn}
\subsection{$PSU(p, q)$ }
\begin{dfn}
	\emph{Проективной псевдо-унитарной группой $PSU(p,\,q)$} называется
	фактор-группа $SU(p,\,q) / \mathbf{Z}_n$, где  $p+q=n$.
\end{dfn}
\subsection{$\mathcal{O}_{J}$ }
$\mathcal{O}_{J} \equiv Sp(2r,\, \CC)$ 
\end{document} % конец документа

