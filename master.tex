
\documentclass[a4paper,12pt]{article}

% В этом документе преамбула

%%% Работа с русским языком
\usepackage{cmap}					% поиск в PDF
\usepackage{mathtext} 				% русские буквы в формулах
\usepackage[T2A]{fontenc}			% кодировка
\usepackage[utf8x]{inputenc}			% кодировка исходного текста
\usepackage[english,russian]{babel}	% локализация и переносы
\usepackage{indentfirst}
\frenchspacing

\renewcommand{\epsilon}{\ensuremath{\varepsilon}}
\renewcommand{\phi}{\ensuremath{\varphi}}
\renewcommand{\kappa}{\ensuremath{\varkappa}}
\renewcommand{\le}{\ensuremath{\leqslant}}
\renewcommand{\leq}{\ensuremath{\leqslant}}
\renewcommand{\ge}{\ensuremath{\geqslant}}
\renewcommand{\geq}{\ensuremath{\geqslant}}
\renewcommand{\emptyset}{\varnothing}

%%% Дополнительная работа с математикой
\usepackage{amsmath,amsfonts,amssymb,amsthm,mathtools} % AMS
\usepackage{icomma} % "Умная" запятая: $0,2$ --- число, $0, 2$ --- перечисление

%% Номера формул
%\mathtoolsset{showonlyrefs=true} % Показывать номера только у тех формул, на которые есть \eqref{} в тексте.
%\usepackage{leqno} % Нумереация формул слева

%% Свои команды
\DeclareMathOperator{\sgn}{\mathop{sgn}}

%% Перенос знаков в формулах (по Львовскому)
\newcommand*{\hm}[1]{#1\nobreak\discretionary{}
{\hbox{$\mathsurround=0pt #1$}}{}}

%%% Работа с картинками
\usepackage{graphicx}  % Для вставки рисунков
\graphicspath{{images/}{images2/}}  % папки с картинками
\setlength\fboxsep{3pt} % Отступ рамки \fbox{} от рисунка
\setlength\fboxrule{1pt} % Толщина линий рамки \fbox{}
\usepackage{wrapfig} % Обтекание рисунков текстом

%%% Работа с таблицами
\usepackage{array,tabularx,tabulary,booktabs} % Дополнительная работа с таблицами
\usepackage{longtable}  % Длинные таблицы
\usepackage{multirow} % Слияние строк в таблице

%%% Теоремы
\theoremstyle{plain} % Это стиль по умолчанию, его можно не переопределять.
\newtheorem{theorem}{Теорема}[section]
\newtheorem{proposition}[theorem]{Утверждение}
 
\theoremstyle{definition}% "Определение"
\newtheorem*{dfn}{Определение}
\newtheorem{corollary}{Следствие}[theorem]
\newtheorem{problem}{Задача}[section]
 
\theoremstyle{remark} % "Примечание"
\newtheorem*{nonum}{Решение}

%%% Программирование
\usepackage{etoolbox} % логические операторы

%%% Страница
\usepackage{extsizes} % Возможность сделать 14-й шрифт
\usepackage{geometry} % Простой способ задавать поля
	\geometry{top=25mm}
	\geometry{bottom=35mm}
	\geometry{left=35mm}
	\geometry{right=20mm}
 %
%\usepackage{fancyhdr} % Колонтитулы
% 	\pagestyle{fancy}
 	%\renewcommand{\headrulewidth}{0pt}  % Толщина линейки, отчеркивающей верхний колонтитул
% 	\lfoot{Нижний левый}
% 	\rfoot{Нижний правый}
% 	\rhead{Верхний правый}
% 	\chead{Верхний в центре}
% 	\lhead{Верхний левый}
%	\cfoot{Нижний в центре} % По умолчанию здесь номер страницы

\usepackage{setspace} % Интерлиньяж
%\onehalfspacing % Интерлиньяж 1.5
%\doublespacing % Интерлиньяж 2
%\singlespacing % Интерлиньяж 1

\usepackage{lastpage} % Узнать, сколько всего страниц в документе.

\usepackage{soul} % Модификаторы начертания

\usepackage{hyperref}
\usepackage[usenames,dvipsnames,svgnames,table,rgb]{xcolor}
\hypersetup{				% Гиперссылки
    unicode=true,           % русские буквы в раздела PDF
    pdftitle={Заголовок},   % Заголовок
    pdfauthor={Автор},      % Автор
    pdfsubject={Тема},      % Тема
    pdfcreator={Создатель}, % Создатель
    pdfproducer={Производитель}, % Производитель
    pdfkeywords={keyword1} {key2} {key3}, % Ключевые слова
    colorlinks=true,       	% false: ссылки в рамках; true: цветные ссылки
    linkcolor=red,          % внутренние ссылки
    citecolor=black,        % на библиографию
    filecolor=magenta,      % на файлы
    urlcolor=cyan           % на URL
}

\usepackage{csquotes} % Еще инструменты для ссылок

%\usepackage[style=apa,maxcitenames=2,backend=biber,sorting=nty]{biblatex}

\usepackage{multicol} % Несколько колонок

\usepackage{tikz} % Работа с графикой
\usepackage{pgfplots}
\usepackage{pgfplotstable}
%\usepackage{coloremoji}
\usepackage{floatrow}
\usepackage{subcaption}
\newcommand*{\N}{\mathbb{N}}
\newcommand*{\R}{\mathbb{R}}
\newcommand*{\K}{\mathbb{K}}
\newcommand*{\V}{\mathcal{V}}
\newcommand*{\A}{\mathcal{A}}
\newcommand*{\ii}{\mathbf{1}}
\newcommand*{\oo}{\mathbf{0}}
\newcommand*{\ba}{\mathbf{a}}
\newcommand*{\bb}{\mathbf{b}}
\newcommand*{\Q}{\mathbb{Q}}
\newcommand*{\CC}{\mathbb{C}}
\graphicspath{{images/}}
%\setcounter{secnumdepth}{0}
%\addbibresource{master.bib}
\usepackage{titlesec}% http://ctan.org/pkg/titlesec
\titleformat{\section}{\normalfont\Large\bfseries}{Часть }{}{\thesection}
%\renewcommand{\thesection}{}% Remove section references\ldots
\renewcommand{\thesubsection}{\arabic{subsection}}%... from subsections
\usepackage{marginnote}



%%% Local Variables:
%%% mode: latex
%%% TeX-master: "master"
%%% End:


%\author{Ярослав😎}
\title{Домашка по теории групп}
%\date{\today}

%\includeonly{chapters/ch2,chapters/ch3}

\begin{document} % конец преамбулы, начало документа

\maketitle

\section{}
\subsection{$\mathcal{U}_\mathbf{H}$ }
\begin{dfn}
	\emph{Группой} $\,\mathcal{U}_\mathbf{H}$ называется группа невырожденных
	комплексных матриц $T$, удовлетворяющих соотношению
	\begin{equation}
		\mathbf{H}=T^\dag \cdot \mathbf{H} \cdot T
		\label{eq:1}
	,\end{equation}
	где $\mathbf{H}$ --- некоторая эрмитова матрица.
\end{dfn}
\subsection{$O(n,\, \CC )$ }
\begin{dfn}
	\emph{Ортогональной группой $O(n,\,\CC)$} называется группа комплексных
	ортогональных
	матриц
	\begin{equation}
	O^T \cdot O = I_n
	\label{eq:2}
.\end{equation}
\end{dfn}

Группа вещественных ортогональных матриц $O(n)$ и  соответствующее
многообразие, вложенное в
$\R^{n^2}$, задаётся системой уравнений (\ref{eq:2}).
Поскольку матрицы $O \cdot O^T$ и $I_n$ --- симметричны, то число
независимых уравнений, определяющих многообразие $O (n)$, равно $n(n+1) /2$.
Отсюда следует, что многообразие группы $O(n)$ имеет размерность
\begin{equation}
	\dim (O(n)) = n^2 - n (n+1) /2=n(n-1) /2 = \dim (SO(n))	
	\label{eq:on}
.\end{equation}
Последнее равенство вытекает из того факта, что многообразие группы $O(n)$ 
состоит из двух несвязанных частей одинаковой размерности, одна из которых
представляет собой многообразие $SO(n)=)_+(n)$ (множество ортогональных матриц
$O$ с условием $\det (O)=1$, а вторая часть $O_-(n)$) (множество ортогональных
матриц $O$ с условием $\det (O)=-1$) состоит из элементов смежного класса
$R \cdot SO(n)$, где $R$ любой фиксированный элемент из $O_-(n)$.
\subsection{$SO(n,\, \CC )$ }
\begin{dfn}
	\emph{Специальной ортогональной группой $SO(n,\,\CC)$} называется
	подгруппа комплексных ортогональных
матриц, удовлетворяющих (\ref{eq:2}), и таких, что $\det (O)=+1$.
\end{dfn}
Это группа Ли размерности (\ref{eq:on}) как подгруппа $O(n)$.
Алгебра Ли группы $SO(2)$ --- $so(2)$. Рассмотрим кривую $O_\phi$ около
единичного элемента (для малых углов  $\phi$)
\begin{equation}
	O_\phi = I_2 +\phi \mathbf{i}+ O(\phi^2), \quad \mathbf{i}=
	\begin{pmatrix} 0 & -1 \\
	1 & 0 \end{pmatrix} 
.\end{equation}
Сравнивая кривую $g(t) = O_\phi \mid _{\phi=at}$ ($t$ --- мало и $a \in  \R$ с
выражением, связывающим элементы алгебры и группы Ли в малой окрестности
\begin{equation}
	g\left( t \right) = I_n +t A + O (t^2)
,\end{equation}
мы заключаем что алгебра Ли $so(2)$ состоит из антисимметричных двумерных
матриц $a \mathbf{i}$, которые образуют одномерное векторное пространство.
Отметим, что любой элемент $O_\phi$ группы $SO(2)$ можно представить в виде
линейной комбинации матриц $I_2$ и $\mathbf{i}$ 
 \begin{equation}
	\begin{pmatrix} 
	\cos \phi & - \sin \phi \\
\sin \phi & \cos \phi \end{pmatrix} =
I_2 \cos \phi +\mathbf{i} \sin \phi
\label{eq:3215}
.\end{equation}
Имеет место следующее тождество (операторный аналог формулы Эйлера):
\begin{equation}
	\exp \left(\hat{\mathbf{i}} \phi\right)= I \cos \phi + \hat{\mathbf{i}}
	\sin\phi, \quad \phi \in \R \text{ или } \CC
	.\label{eq:euler}
\end{equation}
Из формулы (\ref{eq:euler}) следует, что любой элемент $O_\phi \in SO(2)$,
заданный в (\ref{eq:3215}), представляется в виде $O_\phi= \exp (\mathbf{i} \phi)$,
то есть в виде экспоненты от элемента $(\mathbf{i}\phi)$ алгебры Ли $so(2)$.

\marginnote{Настя О.}[0cm]
	\begin{equation}
		 A\in SO(2,\,\CC)\Leftrightarrow (A^{T}=A^{-1},\, \det A=1)
	.\end{equation}
	$SO(n,\,\CC)$ абелева группа, т.к. она изоморфна группе вращений
	$(n-1)$-мерной сферы, которая является абелевой.

	Т.к. $SO(n,\,\CC)$ абелева группа, то $ Z( SO(n,\,\CC))= SO(n,\,\CC) $,
	значит классом эквивалентности любого элемента будет являться тот же
	элемент.

	Т.к. $SO(n,\,\CC)$ изоморфна группе вращений $(n-1)$-мерной сферы,
	сфера изоморфна множеству своих вращений и является гладким
	многообразием, то $SO(n,\,\CC)$ тоже гладкое многообразие (кроме того,
	состоит из бесконечного числа элементов) $ \Rightarrow $  $SO(n,\,\CC)$
	--- группа Ли.

	Рассмотрим матрицу $ A\in SO(n,\,\CC) $ близкую к $I_n$:
	\begin{equation}
		 A=I_n+tB+O(t^{2}),\quad t \to  0 
	.\end{equation}
	Т.к. $ A^{T}=A^{-1} $, то $ I_n=AA^{T}=I_n+t(B+B^{T})+
	O(t^{2})$, значит алгебра Ли $so(2,\,\CC)$ состоит из матриц $B$:
	$ -B=B^{T} $, т.е. кососимметричных.

	$ a_{i},\, i=1,\,2,\,3,\ldots, n(n-1) /2 $ --- генераторы алгебры Ли.

	Структурные константы для $SO(2,\,\CC)$ равны $0$ (т.к. генератор всего
	$1$).

	Структурные константы для $SO(3,\,\CC)$ равны $ \varepsilon_{ij}^{k} $
	(символ Леви-Чевиты).

	Возножно, для других $n$ --- тоже. Не знаю как доказать. Вообще, они
	зависят от выбора генераторов, который неоднозначен, но я брала самый
	простой.
	
	$SO(n,\,\R)$ компактна, т.к. сфера является компактом, а для
	$SO(n,\,\CC)$ нельзя ввести для понятие компактности (оно только для
	вещественных алгебр).
	
	$ g_{ab}=C^{d}_{ac}C^{c}_{bd} $ --- метрика Киллинга
	($ C^{d}_{ac}C^{c}_{bd} $ --- структурные константы).

	$so(2,\,\CC)$ не полупростая, а $so(3,\,\CC)$ полупростая по критерию
	Картана ($ \det(g_{ab})\neq0 $).

	$so(2,\,\CC)$ абелева, а для абелевой алгебры понятия простоты я не
	нашла.

	У $so(3,\,\CC)$ нет неочевидных идеалов, значит она простая.

	В $SO(2,\,\R)$ и в $SO(3,\,\R)$ $ C^1(SO(2,\,\R))=SO(2,\,\R),\;
	C^1(SO(3,\,\R))=SO(3,\,\R)$, значит они нильпотентны.
\subsection{$Sp(2r,\, \R)$ }
\begin{dfn}
	\emph{Симплектической группой $Sp(2r,\, \R)$ } называется группа
	вещественных матриц $T$,
		удовлетворяющих условию симплектичности
	\begin{equation}
		T^T \cdot \begin{pmatrix} 0 & I_{r} \\ -I_{r} & 0 \end{pmatrix} 
		\cdot T= \begin{pmatrix} 0 & I_{r} \\ -I_{r} & 0 \end{pmatrix} 
		\label{eq:3}
	.\end{equation}

\end{dfn}
Алгебра Ли $sp(2r,\R$ группы $Sp(2r,\,\R)$. Для симплектических матриц близких к
единице имеем
\begin{equation}
	J=(I_{2r} +t A^T + O(n^2)) \cdot J \cdot (I_{2r} +t A + O(t^2))=
	J+t(A^T \cdot J + J \cdot A) + O(t^2)
,\end{equation}
то есть
\begin{equation}
	A^T \cdot J + J \cdot A =0
,\end{equation}
где антисимметричная матрица $J$ задана в (\ref{eq:3}). Таким образом,
$sp(2r,\,\R)$ --- это множество матриц, удовлетворяющих условию
\begin{equation}
	A^T \begin{pmatrix} 
0 & I_r \\
-I_r & 0
\end{pmatrix}
+
\begin{pmatrix} 
0 & I_r \\
-I_r & 0\end{pmatrix} A = 0
\label{eq:3226}
.\end{equation}
Матрицу $A$ можно представить в блочном виде
\begin{equation}
	A =
	\begin{pmatrix} 
	X & Y \\
Z & W\end{pmatrix} 
\label{eq:a}
,\end{equation}
где $X,\,Y,\,Z,\,W$ --- матрицы $r \times r$ с элементами из $\R$, которые в
силу условия (\ref{eq:3226}) удовлетворяют соотношениям
 \begin{equation}
	Y^T = Y, \qquad Z^T = Z, \qquad X^T = - W
.\end{equation}
Вещественная размерность пространства $sp(2r,\, \R)$ равна
\begin{equation}
	\dim (sp(2r,\,\R))= r (2r +1)
.\end{equation}
Если $A \in sp(2r,\R)$, то $\exp (A) \in  Sp(2r,\,\R)$.
\subsection{$Sp(p,\, q)$ }
\begin{dfn}
	\emph{ Группой $Sp(p,\,q)$}, где $p+q=r$, называется группа
	комплексных $2r \times 2r$ матриц $T$, удовлетворяющих одновременно
	условию псевдо-унитарности
	\begin{equation}
	T^\dag \cdot \begin{pmatrix} 
	I_{p,\, q} & 0\\
0 & I_{p,\,q}
\end{pmatrix} \cdot T= \begin{pmatrix}
I_{p,\,q} & 0\\
0 & I_{p,q}\end{pmatrix} 
	\end{equation}
	и нестандартному условию симплектичности
	\begin{equation}
		T^T \cdot \begin{pmatrix} 0 & I_{p,\,q} \\ -I_{p,\,q} & 0
		\end{pmatrix} 
		\cdot T= \begin{pmatrix} 0 & I_{p,\,q} \\ -I_{p,\,q} & 0
		\end{pmatrix} 
	.\end{equation}
\end{dfn}
Алгебры Ли $sp(p,\,q)$. Рассматривая элементы группы $Sp(p,\,q)$, близкие к 
единичным, получаем, что алгебра Ли $sp(p,\,q)$ --- это множество $2r \times 2r$ 
(здесь $r = p +  q$) комплексных матриц $A$, удовлетворяющих соотношениям
 \begin{equation}
	 A^T \begin{pmatrix} 0 & I_r \\ -I_r & 0\end{pmatrix} =
	 \begin{pmatrix} 0 & -I_r \\ I_r & 0 \end{pmatrix} A,
	 \quad A^\dag \begin{pmatrix} I_{p,\,q} & 0 \\ 0 & I_{p,\,q} \end{pmatrix}
	 = - \begin{pmatrix} I_{p,\,q} & 0 \\ 0 & I_{p,\,q} \end{pmatrix} 
	 \label{eq:sp}
.\end{equation}
Представим $A$ в виде блочной матрицы (\ref{eq:a}), где $X,\,Y,\,Z,\,W$ ---
комплексные $r \times r$ блоки. Из соотношений (\ref{eq:sp}) следуют условия
 \begin{align}
	 W = - X^T,&\qquad Z=-I_{p,\,q} \cdot Y^\dag \cdot I_{p,\,q},\\
	 Y^T = Y,&\qquad X^\dag = - I_{p,\,q} \cdot X \cdot I_{p,\,q},
	 \label{eq:3234}
\end{align}
и любой элемент $A \in sp(p,\,q)$ представляется в виде блочной матрицы
\begin{equation}
	A= \begin{pmatrix} 
	X & Y \\ -I_{p,\,q} \cdot Y^\dag \cdot I_{p,\,q} & - X^T \end{pmatrix} 
,\end{equation}
где две $r \times r$ матрицы $X$ и $Y$, удовлетворяют соотношениям обобщённой
антиэрмитовости и симметричности (\ref{eq:3234}). Пространства
таких матриц $X $ и $Y$  имеют вещественные размерности $r^2$ и $(r+1)r$,
соответственно. Поэтому
\begin{equation}
	\dim (sp(p,\,r-p))=(r+1)r + r^2= r(2r +1)
	\label{eq:3236}
.\end{equation}
\subsection{$USp(2r)$ }
\begin{dfn}
	\emph{Унитарной симлектической группой $USP(2r)$} называется
	группа комплексных $2r \times 2r$ матриц $T$, удовлетворяющих как условию
	симплектичности  (\ref{eq:3}), так и условию унитарности $T^\dag \cdot
	T=I_n$.
\end{dfn}
Алгебра Ли $usp(2r)$ группы $USp(2r)$. Для унитарных симплектических
$(2r \times 2r$ матриц, близких к единичной, кроме соотношения унитарности
\begin{equation}
	I_n = U U^\dag = (I_n + t A + O(t^2))(I_n+t A^\dag + O(t^2))= I_n+
	t(A +A^\dag) + O(t^2)
,\end{equation}
откуда
\begin{equation}
	A^\dag = A
	\label{eq:uni}
,\end{equation}
имеем ещё условие симплектичности. Поэтому алгебра Ли $usp(2r)$ --- это
множество всех комплексных матриц $A$, одновременно удовлетворяющих
соотношениям (\ref{eq:uni}) и (\ref{eq:3226}), которые можно записать в виде
(\ref{eq:sp}), где вместо матрицы $I_{p,\,q}$ необходимо взять единичную
матрицу $I_r$. Таким образом, $usp(2r)=sp(r,\,0)=sp(0,\,r)$. Теперь мы можем
воспользоваться результатами предыдущего пункта и получить для произвольного
элемента $A \in  usp(2r)$ представление в виде блочной матрицы
 \begin{equation}
	 A = \begin{pmatrix}  X & Y \\ -Y^\dag & -X^T \end{pmatrix} 
	 \label{eq:3237}
,\end{equation}
где $X$ и $Y$ --- комплексные $r \times r$ матрицы, удовлетворяющие условиям
\begin{equation}
	Y^T=Y, \qquad X^\dag=-X
	\label{eq:3238}
.\end{equation}
Легко проверить, что множество матриц $usp(2r)$ (\ref{eq:3237}),
(\ref{eq:3238}) образует векторное пространство, замкнутое относительно
операции коммутирования. Размерность этого пространства равна
\begin{equation}
	\dim (usp(2r))=r(2r+1),
\end{equation} 
что следует из (\ref{eq:3236}).
\subsection{$O(p, q)$ }
\begin{dfn}
	\emph{Вещественной псевдо-ортогонально группой $O(p,\,q)$} называется
	группа вещественных $(n \times n)$ матриц $O$, где $n=p+q$, подчиняющихся
	условию
	\begin{equation}
		O^T \cdot
		\begin{pmatrix} 
		I_p & 0 \\
	0 & -I_q
\end{pmatrix}
\cdot O = 
\begin{pmatrix} 
I_p & 0 \\
0 & -I_q\end{pmatrix} 
	.
	\label{eq:6}
	\end{equation}
\end{dfn}
\subsection{$PSO(p, q)$ }
\begin{dfn}
	\emph{Специальной вещественной псевдо-ортогональной группой $SO(p,\,q)$}
	называется
	группа вещественных $(n \times n)$ матриц $O$, где $n=p+q$, подчиняющихся
	одноврменно условию псевдоортогональности (\ref{eq:6}), а также
	условию $\det O =1$.
\end{dfn}

\begin{dfn}
	\emph{Проективной псевдо-ортогональной группой $PSO(p,\,q)$}
	называется фактор-группа $SO(p,\,q) / \mathbf{Z}_2$.
\end{dfn}
\subsection{Группа Лоренца}
\begin{dfn}
	\emph{Группой Лоренца $n$-мерного пространства}
	называется группа $O(1,n-1)$.
\end{dfn}
\subsection{$SU(p, q)$ }
\begin{dfn}
	\emph{Специальной псевдо унитарной группой $SU(p,\,q)$} называется
	группа комплексных $(n \times n)$ матриц $U$, удовлетворяющих
	одновременно условию (\ref{eq:1}),
	где в качестве $\mathbf{H}$ выбрана матрица $I_{p,\,q}$:
\begin{equation}
	U^\dag  \cdot \begin{pmatrix} I_{p,\,q} & 0 \\ 0 & I_{p,\,q}
	\end{pmatrix} \cdot U= \begin{pmatrix} I_{p,\,q} & 0 \\  0 & I_{p,\,q}
\end{pmatrix} 
\end{equation}
и условию  $\det U =1$.
\end{dfn}
\emph{Центром группы} является конечная инвариантная подгруппа матриц
\begin{equation}
	U_k =\exp \left( i \frac{2 \pi k}{n} \right) I_n \quad (k = 0, 1, \ldots,
	n-1)
.\end{equation}
\subsection{$PSL(n, \K)$ }
\begin{dfn}
	\emph{Проективной комплексной линейной группой $PSL(n,\,\CC)$}
	называется фактор-группа $SL(n,\,\CC) / \mathbf{Z}_n$.
\end{dfn}
\begin{dfn}
	\emph{Проективной вещественной линейной группой $PSL(2n,\,\R$}
	называется фактор-группа $SL(2n\,\R)/\mathbf{Z}_2$.
\end{dfn}
\subsection{$\mathcal{O}_{I_n}$ }
$\mathcal{O}_{I_n} \equiv O(n,\,\CC)$
\subsection{Группа анти-де Ситтера}
\begin{dfn}
	\emph{Группой анти-де Ситтера $n$-мерного пространства} называется
	группа $O(2,n-2)$.
\end{dfn}
\subsection{$PSU(p, q)$ }
\begin{dfn}
	\emph{Проективной псевдо-унитарной группой $PSU(p,\,q)$} называется
	фактор-группа $SU(p,\,q) / \mathbf{Z}_n$, где  $p+q=n$.
\end{dfn}
\subsection{$\mathcal{O}_{J}$ }
$\mathcal{O}_{J} \equiv Sp(2r,\, \CC)$ 
\subsection{Группа Пуанкаре}
\begin{dfn}
	\emph{Группой Пункаре} называется группа симметрий четырёхмерного
	пространства Минковского, т.е. множество преобразований вида
	\begin{equation}
 		x_k \to x'_k = O_{kj}x_j+a_k
 	,\end{equation}
 	где вещественная матрица $\|O_{ij}\| \in O(1,\,3)$ и произведение
 	элементов задаётся следующим образом
 	\begin{equation}
 		g_1 \cdot g_2 = (O_1, \vec{a}_1) \cdot  (O_2, \vec{a}_2) =
 		(O_1 \cdot O_2,\, O_1 \cdot \vec{a}_2 + \vec{a}_1)
 	.\end{equation}
\end{dfn}
\end{document} % конец документа

